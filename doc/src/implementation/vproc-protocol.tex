\documentclass[11pt]{article}

\setlength{\textwidth}{6in}
\setlength{\oddsidemargin}{0.25in}
\setlength{\textheight}{9.0in}
\setlength{\topmargin}{-0.75in}

% common LaTeX macros
%
% Last modified: 03-02-2007
%

\usepackage{times}
%-------------------------
% the following magic makes the tt font in math mode be the same as the
% normal tt font (i.e., Courier)
%
\SetMathAlphabet{\mathtt}{normal}{OT1}{pcr}{n}{n}
\SetMathAlphabet{\mathtt}{bold}{OT1}{pcr}{bx}{n}
%-------------------------

\usepackage{amsmath}
\usepackage{amssymb} % for \pitchfork

\newcommand{\NOTE}[1]{%
  \par\leavevmode\noindent\textbf{[[ #1 ]]}\par\leavevmode\noindent}
\newcommand{\CUT}[1]{}
\newcommand{\SIDENOTE}[1]{%
  \marginpar{\tiny\raggedright{#1}}}

\newcommand{\appref}[1]{Appendix~\ref{#1}}
\newcommand{\chapref}[1]{Chapter~\ref{#1}}
\newcommand{\secref}[1]{Section~\ref{#1}}
\newcommand{\tblref}[1]{Table~\ref{#1}}
\newcommand{\figref}[1]{Figure~\ref{#1}}
\newcommand{\listingref}[1]{Listing~\ref{#1}}
\newcommand{\pref}[1]{{page~\pageref{#1}}}
\newcommand{\defref}[1]{Definition~\ref{#1}}
\newcommand{\ruleref}[1]{Rule~\ref{#1}}

\newcommand{\eg}{{\em e.g.}}
\newcommand{\cf}{{\em cf.}}
\newcommand{\ie}{{\em i.e.}}
\newcommand{\etc}{{\em etc.\/}}
\newcommand{\naive}{na\"{\i}ve}
\newcommand{\ala}{{\em \`{a} la\/}}
\newcommand{\etal}{{\em et al.\/}}
\newcommand{\role}{r\^{o}le}
\newcommand{\vs}{{\em vs.}}
\newcommand{\forte}{{fort\'{e}\/}}

%
% language names
\newcommand{\Cplusplus}{\mbox{C\hspace{-.05em}\raisebox{.4ex}{\tiny\bf ++}}}
\newcommand{\Cmm}{\mbox{C\hspace{-.05em}\raisebox{.4ex}{\small\textbf{{-}{-}}}}}
\newcommand{\csharp}{\textsc{C\#}}
\newcommand{\C}{\textsc{C}}
\newcommand{\Ckit}{\textsc{Ckit}}
\newcommand{\java}{\textsc{Java}}
\newcommand{\loom}{\textsc{Loom}}
\newcommand{\moby}{\textsc{Moby}}
\newcommand{\minimoby}{\textsc{MiniMoby}}
\newcommand{\micromoby}{\textsc{microMoby}}
\newcommand{\MOC}{\textsc{MOC}}
\newcommand{\ml}{\textsc{ML}}
\newcommand{\sml}{\textsc{SML}}
\newcommand{\smlnj}{\textsc{SML/NJ}}
\newcommand{\mlj}{\textsc{MLj}}
\newcommand{\cml}{\textsc{CML}}
\newcommand{\pml}{\textsc{PML}}
\newcommand{\ocaml}{\textsc{OCaml}}
\newcommand{\mlkk}{\textsc{ML2000}}
\newcommand{\haskell}{\textsc{Haskell}}
\newcommand{\mltwok}{\textsc{ML2000}}
\newcommand{\scala}{\textsc{Scala}}
\newcommand{\perl}{\textsc{Perl}}
\newcommand{\scheme}{\textsc{Scheme}}
\newcommand{\unix}{\textsc{Unix}}
\newcommand{\smalltalk}{\textsc{Smalltalk}}
\newcommand{\self}{\textsc{Self}}

%
% font commands
\providecommand{\bftt}[1]{{\ttfamily\bfseries{}#1}}
\providecommand{\ittt}[1]{{\ttfamily\itshape{}#1}}
\providecommand{\kw}[1]{\bftt{#1}}
\providecommand{\nt}[1]{{\rmfamily\itshape{#1}}}
\providecommand{\term}[1]{{\sffamily{#1}}}
%
% math-mode versions
\providecommand{\mkw}[1]{\ensuremath{\text{\kw{#1}}}}
\providecommand{\mnt}[1]{\ensuremath{\text{\nt{#1}}}}
\providecommand{\mterm}[1]{\ensuremath{\text{\term{#1}}}}

% braces (in math mode)
\newcommand{\LCB}{\mkw{\{}}
\newcommand{\RCB}{\mkw{\}}}

% underscore
\newcommand{\US}{\char`\_}

%%%%%
% Some common math notation
%

% double brackets
\newcommand{\LDB}{\ensuremath{[\mskip -3mu [}}
\newcommand{\RDB}{\ensuremath{]\mskip -3mu ]}}

\newcommand{\dom}{\ensuremath{\mathrm{dom}}}
\newcommand{\rng}{\ensuremath{\mathrm{rng}}}

% sets
\newcommand{\SET}[1]{\ensuremath{\{#1\}}}
\newcommand{\Fin}{\textrm{Fin}}     % finite power set
\newcommand{\DISJOINT}[2]{\ensuremath{#1 \pitchfork #2}}
\newcommand{\finsubset}{\mathrel{\stackrel{\textrm{fin}}{\subset}}}


% finite maps
\newcommand{\finmap}{\mathrel{\stackrel{\textrm{fin}}{\rightarrow}}}
\newcommand{\MAP}[2]{\SET{#1 \mapsto #2}}
\newcommand{\EXTEND}[2]{\ensuremath{#1{\pm}#2}}
\newcommand{\EXTENDone}[3]{\EXTEND{#1}{\MAP{#2}{#3}}}
\newcommand{\SUBST}[3]{\ensuremath{#1[#2\mapsto{}#3]}}
\newcommand{\SUBSTTWO}[5]{\ensuremath{#1[#2\mapsto{}#3,#4\mapsto{}#5]}}


% timestamp
\newcount\timeHH
\newcount\timeMM
\timeHH=\time
\divide\timeHH by 60
\timeMM=\time
\count255=\timeHH
\multiply\count255 by -60 \advance\timeMM by \count255
\newcommand{\timestamp}{%
  \today{} ---
  \ifnum\timeHH<10 0\fi\number\timeHH\,:\,\ifnum\timeMM<10 0\fi\number\timeMM}


\usepackage{graphicx}
\usepackage{listings}

\title{VProc protocol}
\author{The Manticore Group}
\date{Draft of \today}

\begin{document}
\maketitle

\section{Overview}
This document presents two alternative messaging protocols for vprocs.
Both protocols have their advantages, and this document is an attempt to understand
which is most appropriate for the Manticore implementation.
The document is structured as follows.
Section \ref{sec:preliminaries} outlines the interface and semantics of the protocol.
Section \ref{sec:protocol1} describes the first protocol, and section
\ref{sec:protocol2} describes the second protocol.
Then \secref{sec:conclusion} compares the protocols.

\section{Preliminaries}\label{sec:preliminaries}

\paragraph{Landing pad}
The \emph{landing pad} is a lock-free linked list of incoming messages.
Each vproc owns a landing pad, and has access to all the other remote landing pads.
The landing pad supports two operations.
\begin{enumerate}
  \item Push a message on a remote landing pad.
  \item Pop all messages from the local landing pad.
\end{enumerate}

\paragraph{Sleeping}
When a vproc has nothing to do, it becomes idle by waiting on a condition variable
provided by the OS.
Once messages arrive on the landing pad the vproc must wake and service those messages.

Below are the two relevant operations for the protocol.
\lstset{language=C}
\lstset{commentstyle=\textit}
\begin{lstlisting}
void Sleep (VProc_t *self);
void Send (VProc_t *vp, Fiber_t msg);
\end{lstlisting}
The Sleep operation puts the vproc to sleep on a condition variable.
The Send operation places a message on a vproc's landing pad.
These operations are related because of the following requirement: a vproc must wake
soon after a message appears on its landing pad.

\section{Protocol 1}\label{sec:protocol1}

\begin{figure}
\lstset{language=C}
\lstset{commentstyle=\textit}
\lstset{numbers=left}
\begin{lstlisting}
void Send (VProc_t *vp, Fiber_t msg)
{
  while (true) {
    Stk_t *stk = vp->lp;
    Stk_t *newStk = Promote(Cons(msg, stk));
    Stk_t *x = CAS(&(vp->stk), stk, newStk);
    if (x != stk) {
      continue;
    } else {
      if (vp->sleeping) {
	MutexLock(&(vp->lock));
          CondSignal(&(vp->signal));
	MutexUnlock(&(vp->lock));
      }
      return;
    }
  }
}
\end{lstlisting}
\caption{Protocol 1 \texttt{Send} operation.}
\end{figure}

\begin{figure}
\lstset{language=C}
\lstset{commentstyle=\textit}
\lstset{numbers=left}
\begin{lstlisting}
void Sleep (VProc_t *self)
{
  MutexLock(&(self->lock));
    self->sleeping = true;
    /* Flush all pending writes */
    while (self->lp == EMPTY)
      CondWait(&(self->cond), &(self->lock));
    self->sleeping = false;
  MutexUnlock(&(self->lock));
}
\end{lstlisting}
\caption{Protocol 1: \texttt{Sleep} operation.}
\end{figure}

\section{Protocol 2}\label{sec:protocol2}

\begin{figure}
\lstset{language=C}
\lstset{commentstyle=\textit}
\lstset{numbers=left}
\begin{lstlisting}
void Send (VProc_t *vp, Fiber_t msg)
{
  while (true) {
    Stk_t *stk = vp->lp;
    if (stk != SLEEPING) {
      Stk_t *newStk = Promote(Cons(msg, stk));
      Stk_t *x = CAS(&(vp->lp), stk, newStk);
      if (x == stk)
	return;
      else
	continue;
    } else {            /* (stk == SLEEPING) */
      Stk_t *newStk = Promote(Cons(msg, EMPTY));
      Stk_t *x = CAS(&(vp->lp), SLEEPING, newStk);
      if (x != SLEEPING)
	continue;
      MutexLock(&(vp->lock));
      if (!vp->flg)
	vp->flg = true;
      else
	CondSignal(&(vp->signal));
      MutexUnlock(&(vp->lock));
      return;
    }
  }
}
\end{lstlisting}
\caption{Protocol 2 \texttt{Send} operation.}
\end{figure}

\begin{figure}
\lstset{language=C}
\lstset{commentstyle=\textit}
\lstset{numbers=left}
\begin{lstlisting}
void Sleep (VProc_t *self)
{
  Stk_t *stk = vp->lp;
  Stk_t *x = CAS(&(vp->stk), EMPTY, SLEEPING);
  if (x == EMPTY) {
    MutexLock(&(vp->lock));
    if (!vp->flg) {
      vp->flg = true;
      CondWait(&(vp->cond), &(vp->lock));
      vp->flg = false;
    }
    MutexUnlock(&(vp->lock));
  }
}
\end{lstlisting}
\caption{Protocol 2: \texttt{Sleep} operation.}
\end{figure}

\section{Conclusion}\label{sec:conclusion}

\CUT{
\begin{figure}[tp]
  \begin{center}
    \includegraphics[scale=0.7]{pictures/vproc-protocol}
  \end{center}%
  \caption{Layout of the VProc protocol.}
  \label{fig:vproc-protocol}
\end{figure}%

\section{Messaging}

\paragraph{\texttt{Send(vp, msg)}}

\begin{description}
\item [\textit{Precondition:}] \textit{It is the case that} \texttt{vp} $\neq$ \texttt{host\_vproc} \textit{(a vproc cannot send a message to itself)}
\end{description}

\begin{enumerate}
  \item Walk to \texttt{vp}'s desk and open the mailbox.
    \begin{enumerate}
      \item If there is a \texttt{SLEEPING} message, place \texttt{msg} in the mailbox and go to step 2.
      \item Otherwise, place \texttt{msg} in the mailbox and exit the subroutine.
    \end{enumerate}
  \item Walk into \texttt{vp}'s \textbf{waiting room}.
    \begin{enumerate}
      \item If the waiting-room switch is off, then do the following. \textit{(\textbf{Invariant:}} \texttt{vp} \textit{is on its way to the waiting room.)}
        \begin{enumerate}
          \item Flip the waiting-room switch on.
          \item Return to the local desk.
          \item Exit the subroutine.
        \end{enumerate}
      \item If the waiting-room switch is on, then go to step 3. \textit{(\textbf{Invariant:}} \texttt{vp} \textit{is in its bedroom.)}
    \end{enumerate}
  \item Flip the bedroom switch on.
  \item Return to the local desk.
\end{enumerate}

\begin{description}
\item [\textit{Postcondition:}] \textit{At some point in the future,} \texttt{vp} \textit{will sit at its desk and open} \texttt{msg}.
\end{description}

\paragraph{\texttt{CheckMailbox()}}

\begin{enumerate}
  \item Open the local mailbox.
  \item Retrieve all messages, leaving the mailbox empty.
  \item Handle each message \texttt{msg}.
\end{enumerate}

\section{Sleeping}

\paragraph{\texttt{Sleep()}}

\begin{enumerate}
  \item Walk to the desk and open the mailbox.
    \begin{itemize}
      \item If empty, place a \texttt{SLEEPING} message in the mailbox.
      \item If nonempty, exit the subroutine.
    \end{itemize}
  \item Walk into the \textbf{waiting room}.
    \begin{itemize}
      \item If the waiting-room switch is on, then go to step 7. \textit{(\textbf{Invariant:}} \textit{Another vproc has delivered mail and has just walked out of the \textbf{waiting room}.)}
      \item If the waiting-room switch is off, then go to step 3. \textit{(\textbf{Invariant:}} \textit{It is the case that, in the time between the current step and step 1, no other vproc has entered the \textbf{waiting room}.)}
    \end{itemize}
  \item Flip the bedroom switch off.
  \item Walk into \textbf{bedroom}.
  \item Wait for the light to turn on.
  \item Walk into \textbf{waiting room}.
  \item Flip the waiting-room switch off.
  \item Return to the desk.
\end{enumerate}

\paragraph{\texttt{Wake(vp)}}

\begin{enumerate}
  \item Construct a blank message \texttt{msg}.
  \item Apply \texttt{Send(vp, msg)} and, once complete, exit the subroutine.
\end{enumerate}

\section{Global GC}

\paragraph{\texttt{GlobalLimit()}}

\begin{enumerate}
  \item Walk into \textbf{leader vproc room}.
    \begin{itemize}
      \item If switch is off, then we assign this vproc as the lead.
        \begin{enumerate}
          \item Flip the global-gc switch on.
          \item Reset turnstyles.
          \item Leave the \textbf{leader vproc room}.
          \item Go to \texttt{LeaderVProc()}.
        \end{enumerate}
      \item If switch is on, then go to step 2.
    \end{itemize}
  \item Walk into \textbf{global gc waiting room}.
  \item Wait for light to turn on.
  \item Enter \textbf{global gc room}.
  \item When finished with global collection, walk to \textbf{global gc departing room}.
  \item Wait for light to turn on.
  \item Return to the desk.
\end{enumerate}

\paragraph{\texttt{LeaderVProc()}}

\begin{enumerate}
  \item For each vproc \texttt{vp}, apply \texttt{Wake(vp)}.
  \item Walk into \textbf{global gc waiting room}.
  \item Wait for light to turn on.
  \item Walk into \textbf{global gc room}.
  \item When finished with global collection, walk to \textbf{global gc departing room}.
  \item Flip the switch off.
  \item Wait for light to turn on.
  \item Return to the desk.
\end{enumerate}
}

\end{document}  
